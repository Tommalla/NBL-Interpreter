\documentclass[a4paper,10pt]{scrartcl}
\usepackage{polski}
\usepackage{ucs}
\usepackage[utf8]{inputenc}
\usepackage{listings}
\lstset{language=C++,inputencoding=utf8x, 
    extendedchars=\true,
    literate={ą}{{\k{a}}}1
             {Ą}{{\k{A}}}1
             {ę}{{\k{e}}}1
             {Ę}{{\k{E}}}1
             {ó}{{\'o}}1
             {Ó}{{\'O}}1
             {ś}{{\'s}}1
             {Ś}{{\'S}}1
             {ł}{{\l{}}}1
             {Ł}{{\L{}}}1
             {ż}{{\.z}}1
             {Ż}{{\.Z}}1
             {ź}{{\'z}}1
             {Ź}{{\'Z}}1
             {ć}{{\'c}}1
             {Ć}{{\'C}}1
             {ń}{{\'n}}1
             {Ń}{{\'N}}1}

%opening
\title{New Better Language (NBL)}
\subtitle{Opis języka}
\author{Tomasz Zakrzewski, tz336079}

\begin{document}

\maketitle

\section{Skrócony opis}
Głównym założeniem NBL jest być językiem podobnym do C++, ale wygodniejszym w użyciu. Podczas projektowania języka posiłkowałem się moją wiedzą
na temat C++ z uwzględnieniem najnowszego standardu C++14 i spróbowałem stworzyć C++ w duchu ``czternastki'', ale nieobciążonego kompatybilnością
wsteczną. W wielu miejscach języka można spotkać pomysły pożyczone z języków funkcyjnych - lambdy, wbudowany typ tablicowy, list comprehensions
(na tablicach) itd. Dla zachowania spójności będę się starał znaleźć kompromis między łatwością wnioskowania o tych elementach właściwą jezykom
funkcyjnym a sposobem działania oczekiwanym po języku imperatywnym.

Ten dokument opisuje mocno założenia, ale sam nie jestem pewien jeszcze w 100\% konkretnej syntaksy, której chcę do realizacji tych funkcjonalności
użyć (upewnię się zapewne w trakcie implementowania, ponieważ będę miał trochę więcej czasu na przemyślenie użyteczności, spójności oraz limitów
wybranych rozwiązań). Z tego powodu mam gorącą prośbę, aby na składnię, szczególnie moich autorskich pomysłów w tym języku patrzeć trochę mniej
wiążąco niż na realizowaną przezeń funkcjonalność. Bardzo mile widziane będą przeze mnie też wszelkie uwagi na temat tego jak pewne 
nieintuicyjne/niespójne elementy składni poprawić.

\section{Standardowe konstrukcje}
Do paczki z tym plikiem załączona jest gramatyka w notacji EBNF, opisująca znaczną część języka. Pozostałe (nietrywialne lub bardzo niestandardowe)
konstrukcje wyjaśniam w tej sekcji.

Wbudowane w język typy to: Void, Int, Double, Char, String, Bool, wskaźnik, krotka (Tuple) oraz tablica (Array). Więcej o tablicach piszę w osobnej poświęconej im sekcji.

NBL posiada wsparcie dla obiektowości.

\subsection{Wskaźniki}
W odróżnieniu od C++, NBL nie posiada referencji. Dostępne są tylko zmienne oraz wskaźniki, przy czym programista nie dostaje dostępu do adresu
wskazywanego przez wskaźnik. Wskaźniki są tylko abstrakcyjnymi dowiązaniami do jakichś isniejących zmiennych. Z języka usunięty jest też operator
brania adresu (\&). Ze standardowych operatorów z C++ pozostaje jedynie dereferencja (*). Przypisanie zmiennych wskaźnikom odbywa się bez żadnych
dodatkowych operatorów. Interpreter jest w stanie sam wywnioskować oczekiwany typ i spróbuje odpowiednio ustawić wskaźnik. Wskaźniki mogą być brane
jedynie od lvalue.

Uwaga notacyjna: W NBL wskaźnik jest częścią typu więc w dobrym guście uważam zapisywanie * przy typie zmiennej, na którą wskazuje wskaźnik, ale
Interpreter ma pozwalać na dowolne ilości spacji między typem, gwiazdką i nazwą wskaźnika.

\begin{lstlisting}
int x = 5;
int y = 10;
int* p1 = x;
int* p2 = y;
IO.stdout.writeln(``%d %d'' % (*p1, *p2)); // 5 6
p1 = y;
p2 = p1; // Równoważne p2 = y lub p2 = *p1;
IO.stdout.writeln(``%d %d'' % (*p1, *p2)); // 6 6
int *p = p1 + p2; // Błąd interpretera 
                  //(wartość p1 jest niezdefiniowana).
int *p = *p1 + *p2; // Błąd interpretera 
                    // (*p1 + *p2 jest rvalue).
\end{lstlisting}

Każdy wskaźnik może też przyjąć specjalną wartość nullptr. W szczególności dla dowolnej zmiennej nullptr spełnia:
\begin{lstlisting}
nullptr == nullptr && nullptr != zmienna
\end{lstlisting}

Przy takim modelu, programista ma duży problem, aby ``strzelić sobie w stopę'' (czyli jak w Javie), ale nie jest pozbawiony podstawowego narzędzia,
jakim są wskaźniki (uff, jednak nie jak w Javie!). W szczególności implementacja swap w NBL wygląda następująco:
\begin{lstlisting}
swap T (T* a, T* b) {
  T c = *a;
  *a = *b;
  *b = c; 
}
\end{lstlisting}

W finalnej wersji będę jeszcze rozważał jakąś formę operatora new. Nie będzie delete, pamięć będzie zwalniana do interpretera w momencie, kiedy
ostatni wskaźnik na dane zniknie ze scope'a.

Do wyłuskiwania pól/metod obiektu pod wskaźnikiem służy znany z C++ operator -> .

Aby uzyskać takie działanie języka, musiałem jednak zrezygnować ze wskaźników wyższego poziomu niż 1. Dzięki temu wnioskowanie o typach przez 
interpreter jest jednoznaczne (w sytuacji, gdy interpretej oczekuje wskaźnika i dostanie wskaźnik, nie istnieje pytanie, czy należy wziąć adres,
czy użyć adresu spod wskaźnika).

\subsection{Krotka}
Mutowalny (w sensie wartości przechowywanych pod współrzędnymi) typ. Krotka danej długości danych typów sama stanowi typ, co oznacza, że nie można
np. przypisać parze intów pary stringów (chyba, że w finalnym języku dam radę zrobić implicit konwersje i odpowiednia konwersja będzie istniała).

Składnia - do przemyślenia. Na pewno będzie się dało robić akcesory jak w tablicy ([numer]).

\subsection{Tablice}
(Array, zachowuje się jak tablica, ale wiele operacji na niej można wykonać
używając składni dla list w np. Haskellu). Implementacja tablic ma pozwalać przynajmniej na używanie jej jak std::vector z C++ (zamortyzowany stały
czas rozszerzania tablicy ``w prawo''), jeśli wystarczy czasu chciałbym też aby dało się szybko dodawać i usuwać elementy z przodu (jak w deque).

Tablica jest typem szablonowym, może przechowywać dowolny typ, np.:
\begin{lstlisting}
Array int arr;
arr.push_back(2);
arr += [3, 4];
arr = [i | i <- [0..1]] + arr;
IO.stdout.writeln(``%s'', arr);	// Wypisuje [0, 1, 2, 3, 4]
\end{lstlisting}

\subsection{Obiektowość}
Klasy, dziedziczenie. Istnieje tylko widoczność publiczna i chroniona (protected). 

\subsubsection{Uniformowy sposób dostępu do składowych}
Język zapewnia automatyczne gettery i settery do publicznych atrybutów klas. Wszelkie atrybuty, do których robilibyśmy settery i gettery w C++, a
same atrybuty byłyby prywatne, w NBL powinny być publiczne i korzystać z uniformowego dostępu do składowych.

Każda klasa może posiadać label modifiers:, poniżej którego występują modyfikatory w formacie:
\begin{lstlisting}
nazwa_pola(typ nowa_wartość) {
  // kod własny
}
\end{lstlisting}

oraz akcesory (accessors):
\begin{lstlisting}
typ_zwracany nazwa_pola {
  kod własny
}
\end{lstlisting}

Dla zilustrowania działania, załóżmy na chwilę, że mamy dwie klasy, A oraz B. Obie mają pole int x, przy czym A ma zdefiniowany modyfikator i
akcesor dla x i obie te metody wypisują na konsolę aktualną wartość x z wykrzyknikiem na końcu. Wtedy poniższy kod ilustruje dobrze działanie
tego mechanizmu:
\begin{lstlisting}
A a; // Deklarujemy obiekt a klasy A.
B b; // Dekl. obiekt b klasy B.
a.x = 5; // Wywołany zostanie modyfikator x z argumentem 5.
b.x = 6;
IO.stdout.writeln(``%d %d'', a.x, b.x); 
// Zauważmy, że wywoływany jest akcesor x z A.
\end{lstlisting}

Powyższy kod wypisze:
\begin{lstlisting}
5!
5!
5 6
\end{lstlisting}

\subsubsection{Rozpakowywanie i wiązanie identyfikatorów}
Do języka chcę wprowadzić popularną w językach funkcyjnych konstrukcję let ... in, która będzie pełniła 2 funkcje:
- analogiczną do {} w C++ lub with w Pythonie - ograniczała życie jakiejś zmiennej tylko do danego bloku kodu (przy okazji)
- rozpakowywanie obiektów.

O ile pierwsza funkcjonalność jest dobrze znana i dosyć prosta, to ta druga jest dosyć ciekawa. Będzie można oczywiście rozpakować w ten sposób
krotki. Ponadto, będzie można rozpakowywać obiekty klas - interpreter spróbuje wtedy dopasować do kolejnych rozpakowywanych elementów kolejne pola
publiczne klasy. Da się rozpakowywać ``przez wskaźnik''.

Weźmy na przykład:
\begin{lstlisting}
class A {
public:
  int x;
  int y;
protected:
  int z;
accessors:
  int x {
    IO.stdout.writeln("x");
    return x;
  }
  
  int y {
    IO.stdout.writeln("y");
    return y;
  }
modifiers:
  x (int x) {
    IO.stdout.writeln("mod x");
    this->x = x;
  }
  
  y (int y) {
    IO.stdout.writeln("mod y");
    this->y = y;
  }
}

int main() {
  A a;
  a.x = 5;
  a.y = 6;
  let x, *y = a in {
    IO.stdout.writeln("%d %d" % (x, *y));
    x = 2;
    *y = 3;
  }
  
  IO.stdout.writeln("%d %d" % (A.x, A.y));
  
  let _, _, z = a in {
    /* Nieważne co jest tutaj, linijka wyżej
       spowoduje błąd interpretacji 
       (dostęp do chronionego pola). */
  }
}
\end{lstlisting}
Powyższy kod, poza błędem interpretacji na końcu, wypisze:

\begin{lstlisting}
x
y
5 6
mod y
x
y
5 3
\end{lstlisting}

\subsection{Operatory}
Wszystkie operatory poza [] oraz () są w NBL binarne. Dwa wspomniane ``wyjątkowe'' operatory mogą przyjmować dowolną stałą ilość argumentów.
Jeśli chodzi o samą składnię, to roboczo przyjmuję chwilowo taką samą składnię jak w C++, z drobną zmianą w postaci opisanych poniżej domyślnych
operatorów:
\begin{lstlisting}
 typ1 operator<op> (typ2 operand1, typ3 operand2) = default;
 // Lub:
 typ1 operator<op> (typ2 operand2) = default; 
 /* Pierwszym operandem w tym przypadku jest obiekt 
 na którego rzecz wywołany jest ten operator. Ten 
 typ deklaracji operatora możliwy jest tylko 
 wewnątrz klasy. */
\end{lstlisting}

\subsubsection{Dla obiektów klas}
NBL potrafi wygenerować domyślne wersje operatorów binarnych, jeśli dla klasy zadeklarowana jest taka intencja. Domyślny operator działa w następujący
sposób: Tworzony zostaje nowy obiekt klasy i jego wewnętrzne pola wypełniane są wynikiem zaaplikowania operatora do odpowiednich wartości pól obu
operandów. Np., mając klasę A z polami int x, int y oraz zadeklarowanym domyślnym operatorem dodawania:

\begin{lstlisting}
  A a;
  A.x = 5;
  A.y = -1;
  A b;
  b.x = 3;
  b.y = 1;
  A c = a + b;
  IO.stdout.writeln("%d %d" % (c.x, c.y));
\end{lstlisting}

Powyższy kod wypisze 8 0.

\subsubsection{Ogólnie}
Operatory (poza () oraz []) można deklarować też poza klasami. Ich operandami mogą być obiekty klas jak i typy wbudowane. W szczególności, można np.
przedefioniować operator + dla dwóch intów. Wartość zwracana może być dowolna, najwyżej nastąpi błąd interpretacji wyrażenia (np. dodawanie inta i
double'a może zwracać stringa).

\end{document}
