\documentclass[a4paper,10pt]{scrartcl}
\usepackage{polski}
\usepackage[utf8]{inputenc}
\usepackage{listings}
\lstset{language=C++}

%opening
\title{New Better Language (NBL)}
\subtitle{Opis języka}
\author{Tomasz Zakrzewski, tz336079}

\begin{document}

\maketitle

\section{Skrócony opis}
Głównym założeniem NBL jest być językiem podobnym do C++, ale wygodniejszym w użyciu. Podczas projektowania języka posiłkowałem się moją wiedzą
na temat C++ z uwzględnieniem najnowszego standardu C++14 i spróbowałem stworzyć C++ w duchu ``czternastki'', ale nieobciążonego kompatybilnością
wsteczną. W wielu miejscach języka można spotkać pomysły pożyczone z języków funkcyjnych - lambdy, wbudowany typ tablicowy, list comprehensions
(na tablicach) itd. Dla zachowania spójności będę się starał znaleźć kompromis między łatwością wnioskowania o tych elementach właściwą jezykom funkcyjnym a sposobem
działania oczekiwanym po języku imperatywnym.

Ten dokument opisuje mocno założenia, ale sam nie jestem pewien jeszcze w 100\% konkretnej syntaksy, której chcę do realizacji tych funkcjonalności
użyć (upewnię się zapewne w trakcie implementowania, ponieważ będę miał trochę więcej czasu na przemyślenie użyteczności, spójności oraz limitów
wybranych rozwiązań). Z tego powodu mam gorącą prośbę, aby na składnię, szczególnie moich autorskich pomysłów w tym języku patrzeć trochę mniej
wiążąco niż na realizowaną przezeń funkcjonalność. Bardzo mile widziane będą przeze mnie też wszelkie uwagi na temat tego jak pewne nieintuicyjne/
niespójne elementy składni poprawić.

\section{Standardowe konstrukcje}
Do paczki z tym plikiem załączona jest gramatyka w notacji EBNF, opisująca znaczną część języka. Pozostałe (trywialne lub bardzo niestandardowe)
konstrukcje wyjaśniam w tej sekcji.

Wbudowane w język typy to: int, char, string, bool, wskaźnik, krotka oraz tablica. Więcej o tablicach piszę w osobnej poświęconej im sekcji.

NBL posiada wsparcie dla obiektowości.

\subsection{Wskaźniki}
W odróżnieniu od C++, NBL nie posiada referencji. Dostępne są tylko zmienne oraz wskaźniki, przy czym programista nie dostaje dostępu do adresu
wskazywanego przez wskaźnik. Wskaźniki są tylko abstrakcyjnymi dowiązaniami do jakichś isniejących zmiennych. Z języka usunięty jest też operator
brania adresu (\&). Ze standardowych operatorów z C++ pozostaje jedynie dereferencja (*). Przypisanie zmiennych wskaźnikom odbywa się bez żadnych
dodatkowych operatorów. Interpreter jest w stanie sam wywnioskować oczekiwany typ i spróbuje odpowiednio ustawić wskaźnik. Wskaźniki mogą być brane
jedynie od lvalue.

Uwaga notacyjna: W NBL wskaźnik jest częścią typu więc w dobrym guście uważam zapisywanie * przy typie zmiennej, na którą wskazuje wskaźnik, ale
Interpreter ma pozwalać na dowolne ilości spacji między typem, gwiazdką i nazwą wskaźnika.

\begin{lstlisting}
int x = 5;
int y = 10;
int* p1 = x;
int* p2 = y;
IO.stdout.writeln(``%d %d'' % (*p1, *p2)); // 5 6
p1 = y;
p2 = p1; // Równoważne p2 = y lub p2 = *p1;
IO.stdout.writeln(``%d %d'' % (*p1, *p2)); // 6 6
int *p = p1 + p2; // Błąd interpretera 
                  //(wartość p1 jest niezdefiniowana).
int *p = *p1 + *p2; // Błąd interpretera 
                    // (*p1 + *p2 jest rvalue).
\end{lstlisting}

Każdy wskaźnik może też przyjąć specjalną wartość nullptr. W szczególności dla dowolnej zmiennej nullptr spełnia:
\begin{lstlisting}
nullptr == nullptr && nullptr != zmienna
\end{lstlisting}

Przy takim modelu, programista ma duży problem, aby ``strzelić sobie w stopę'' (czyli jak w Javie), ale nie jest pozbawiony podstawowego narzędzia,
jakim są wskaźniki (uff, jednak nie jak w Javie!). W szczególności implementacja swap w NBL wygląda następująco:
\begin{lstlisting}
swap T (T* a, T* b) {
  T c = *a;
  *a = *b;
  *b = c; 
}
\end{lstlisting}

W finalnej wersji będę jeszcze rozważał jakąś formę operatora new. Nie będzie delete, pamięć będzie zwalniana do interpretera w momencie, kiedy
ostatni wskaźnik na dane dane zniknie ze scope'a.

\subsection{Krotka}
Mutowalny (w sensie wartości przechowywanych pod współrzędnymi) typ. Krotka danej długości danych typów sama stanowi typ, co oznacza, że nie można
np. przypisać parze intów pary stringów (chyba, że w finalnym języku dam radę zrobić implicit konwersje i odpowiednia konwersja będzie istniała).

Składnia - do przemyślenia. Na pewno będzie się dało robić akcesory jak w tablicy ([numer]).

\subsection{Tablice}
(Array, zachowuje się jak tablica, ale wiele operacji na niej można wykonać
używając składni dla list w np. Haskellu). Implementacja tablic ma pozwalać przynajmniej na używanie jej jak std::vector z C++ (zamortyzowany stały
czas rozszerzania tablicy ``w prawo''), jeśli wystarczy czasu chciałbym też aby dało się szybko dodawać i usuwać elementy z przodu (jak w deque).

Tablica jest typem szablonowym, może przechowywać dowolny typ, np.:
\begin{lstlisting}
Array int arr;
arr.push_back(2);
arr += [3, 4];
arr = [i | i <- [0..1]] + arr;
IO.stdout.writeln(``%s'', arr);	// Wypisuje [0, 1, 2, 3, 4]
\end{lstlisting}

\subsection{Obiektowość}
Klasy, dziedziczenie. Istnieje tylko widoczność publiczna i chroniona (protected). 

\subsubsection{Uniformowy sposób dostępu do składowych}
Język zapewnia automatyczne gettery i settery do publicznych atrybutów klas. Wszelkie atrybuty, do których robilibyśmy settery i gettery w C++, a
same atrybuty byłyby prywatne, w NBL powinny być publiczne i korzystać z uniformowego dostępu do składowych.

Każda klasa może posiadać label modifiers:, poniżej którego występują modyfikatory w formacie:
\begin{lstlisting}
nazwa_pola(typ nowa_wartość) {
  // kod własny
}
\end{lstlisting}

oraz akcesory (accessors):
\begin{lstlisting}
typ_zwracany nazwa_pola {
  kod własny
}
\end{lstlisting}

Dla zilustrowania działania, załóżmy na chwilę, że mamy dwie klasy, A oraz B. Obie mają pole int x, przy czym A ma zdefiniowany modyfikator i
akcesor dla x i obie te metody wypisują na konsolę aktualną wartość x z wykrzyknikiem na końcu. Wtedy poniższy kod ilustruje dobrze działanie
tego mechanizmu:
\begin{lstlisting}
A a; // Deklarujemy obiekt a klasy A.
B b; // Dekl. obiekt b klasy B.
a.x = 5; // Wywołany zostanie modyfikator x z argumentem 5.
b.x = 6;
IO.stdout.writeln(``%d %d'', a.x, b.x); 
// Zauważmy, że wywoływany jest akcesor x z A.
\end{lstlisting}

Powyższy kod wypisze:
\begin{lstlisting}
5!
5!
5 6
\end{lstlisting}


\end{document}
